\documentclass[thesis.tex]{subfile}
\begin{document}
\chapter{Background} \label{ch:Background}
This chapter discusses the literature surrounding robotic state estimation, secure state estimation, and distributed state estimation systems.

\section{State Estimation}
As discussed previously, state estimation is a fundamental problem for robotics. Without knowing where it is, a robot is not very useful. A common technique for robot localization is to use a \gls{kf}. The \gls{kf} is one of the most studied and most heavily utilized Bayesian filters~\cite[39-81]{ProbabilisticRobotics}. %At its core, a \gls{kf} represents the state of a linear system as a multivariate normal distribution. By representing the state of the system as a normal distribution the filter can compactly represent the degree of belief of any possible state of the system.

The actual math of the \gls{kf} is outside the scope of this thesis, and we do not make any changes to the algorithm. The application of the \gls{kf} to the task of robot localization has been studied intently. \textcite{Localization2003, Mohsin2014} both explore details of such applications. The \gls{kf} is popular for three main reasons. First, it is optimal under certain assumptions. Second, it is recursive which is memory efficient. Finally it relies only on having knowledge of noisy sensor data and does not measure certain state variables directly~\cite{Mohsin2014}.

However, the \gls{kf} is best used for a linear system. Thus, plain \glspl{kf} are not commonly used for localization. The two most popular options for real-world localization are the \gls{ekf} and \gls{ukf} which both extend the \gls{kf} to better handle non-linearity. The \gls{ukf} was deveoped by~\textcite{Julier1997}. The \gls{ukf} has two primary advantages over the \gls{ekf}. First, it is more accurate than the \gls{ekf} and this accuracy difference increases as the system becomes less linear. Second, it is computationally less complex than the \gls{ekf}. Both of these contribute to our stated goals in \cref{sec:Problem Statement}.

A \gls{ukf} was chosen to be implemented for this thesis, however many of the statements here regarding the filters can be applied to \glspl{ekf} as well. Our chosen software package also makes it trivially easy to switch between a \gls{ukf} and \gls{ukf}. For most of this thesis we use the generic term \glspl{kf}, demonstrating the fact that these statements apply to both \glspl{ukf} and \glspl{ekf}.

%TODO Should this be re-included?
%\section{Secure State Estimation}
%There are three core goals for the security of any cyber-physical system: integrity, availability, and confidentiality~\cite{Cardenas2008}. Those can be thought of as three simple questions: ``Can I trust the data?", ``Do I have access to the data when I need it?", and ``Is the data hidden from those who shouldn't have it?". This thesis focuses on the first two questions. The third, confidentiality, has less to do with the state estimation system and more to do with the security of the computer and communications systems as a whole.
%
%\subsection{Integrity}
%The accuracy of a \gls{kf} when the sensors are functioning properly is good, so we must look at situations where the sensors are not functioning properly. This leads to two possible scenarios, sensor failure and false data injection.
%
%In a sensor failure scenario one or more of the sensors will stop providing data. This is something that \glspl{kf} can handle without problem. If 100\% of sensors fail, the filter will become non-functional, but as long as at least one sensor remains operational the filter will use whatever information it has available. It is also worth nothing that, due to the design of the filter, the \gls{kf} does not have issues with sensors going in and out of a failure state. When information is available it is used, and when it is not available the filter continues to operate with whatever other information it has.
%
%Knowing that sensor failure is not a huge risk for the \gls{kf}, sensor interference is a much more complex problem. If an attacker is somehow able to inject false data into the filter the pose estimate can be skewed. This is a very real problem. The number of ways that false data could be added to a filter are too numerous to count, ranging from physically blocking a lidar sensor to hijacking the network packets of a wirelessly communicating sensor.
%
%\textcite{Mo2010, Yang2013} both demonstrate that \glspl{kf} are susceptible to false data injection attacks, and that even with failure detectors a clever adversary can craft their attack to bypass these detectors. \textcite{Bezzo_2014, Mo2014} both address this problem by adding extra steps into their \gls{kf} algorithm. Both successfully show that the filter output can be shielded from the effects of the attack.
%
%\subsection{Availability}
%In a distributed system, \gls{DoS} attacks become a real threat. While there are many ways to execute such attacks and also many ways to defeat them~\cite{wood2002denial, bellardo2003802}, such systems are outside the scope of this thesis. Rather than showing that the system can defeat \gls{DoS} attacks, we will show that it can continue to function if external communications are blocked for any reason.

\section{Distributed State Estimation}
As wireless networks become more and more ubiquitous, and given the increasing number and decreasing costs of mobile robots, distributed state estimation systems are desirable. Distributed systems offer a number of advantages that traditional, centralized systems cannot match. These include lower hardware cost and increased resilience to sensor failure, among others.

Traditionally, in order to increase the accuracy of state estimation, more sensors are added to a robot. However, increasing the number of sensors on a robot increases the cost and complexity. If robots can share sensor data with other robots in the surrounding environment, the number of sensors per robot can be decreased without sacrificing state estimate accuracy.

Most research into such distributed systems relies on the concept of cooperative positioning. That is, the robots coordinate their motions in order to use their sensors to determine the state of the other robots. Most often this involves one robot acting as a stationary landmark while the other robot moves.  There is a large body of work exploring the idea of cooperative positioning~\cite{Kurazume1994, Kurazume1996, Kurazume1998, Kurazume2000}. This is a valuable and effective concept when working with a team of coordinated robots. However, assuming close cooperation between all robots in an environment is unrealistic.

Other work into distributed systems still makes the assumption of a cooperative team. \textcite{Sanderson1997, Roumeliotis2002} both demonstrate a system where a \gls{kf} is implemented that estimates the state of multiple robots in a system. However, these states are computed relative to each other, considering the network of robots to be essentially one large robot with different parts. This is also incompatible with our goals, because it makes handling the addition and removal of robots from the environment awkward if not impossible.

The previously mentioned works are all examples of a decentralized, rather than distributed \gls{kf} problem. \textcite{Olfati-Saber2005} makes clear the distinction between these two, explaining that decentralized solutions require all robots to be connected at all times, whereas a distributed system requires only sparse connections between robots. This sparse connection stipulation is much more compatible with our proposed system where robots enter and leave the environment at will. One solution to the distributed problem uses consensus filters and multiple \glspl{kf}~\cite{Olfati-Saber2005}. However, this still requires adaptation of the \glspl{kf} and extra computation, which is something we are trying to avoid.

\end{document}