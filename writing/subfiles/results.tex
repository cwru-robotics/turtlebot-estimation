\documentclass[thesis.tex]{subfile}
\begin{document}
\chapter{Results} \label{Results}
\section{Control Group}
First, we must verify that our localization system works well in Gazebo's noiseless environment. To do this, we examine the \gls{con_filter} state estimate error, specifically the horizontal distance and yaw error, for both a stationary robot and a randomly moving robot.

\subsection{Stationary Robot}

% Table created by stargazer v.5.2 by Marek Hlavac, Harvard University. E-mail: hlavac at fas.harvard.edu
% Date and time: Thu, Aug 04, 2016 - 09:34:50 PM
\begin{table}[h] \centering 
  \caption{Continuous Filter Estimate for Stationary Robot with Noiseless Odometry} 
  \label{tab:continuous_stationary_noiseless_summary} 
\begin{tabular}{@{\extracolsep{5pt}}lccccc} 
\\[-1.8ex]\hline 
\hline \\[-1.8ex] 
Statistic & \multicolumn{1}{c}{N} & \multicolumn{1}{c}{Mean} & \multicolumn{1}{c}{St. Dev.} & \multicolumn{1}{c}{Min} & \multicolumn{1}{c}{Max} \\ 
\hline \\[-1.8ex] 
x\_position & 15,713 & 0.008 & 0.004 & 0.00001 & 0.015 \\ 
y\_position & 15,713 & 0.00003 & 0.00002 & 0.000 & 0.0001 \\ 
yaw & 15,713 & 0.005 & 0.003 & 0.0001 & 0.010 \\ 
x\_error & 15,713 & 0.0000002 & 0.0000004 & $-$0.0000002 & 0.000002 \\ 
y\_error & 15,713 & 0.000 & 0.000 & $-$0 & 0 \\ 
yaw\_error & 15,713 & 0.0000001 & 0.00001 & $-$0.00003 & 0.00004 \\ 
horizontal\_error & 15,713 & 0.0000002 & 0.0000004 & 0.000 & 0.000002 \\ 
\hline \\[-1.8ex] 
\end{tabular} 
\end{table} 


\begin{figure}
\centering
\includegraphics[width=\textwidth, keepaspectratio]{noiseless_continuous_error}
\caption{Error of \gls{con_filter} of Stationary Robot with Noiseless Odometry Over Time}
\label{fig:noiseless_continuous_error}

For a stationary robot in a noiseless environment we can see in \autoref{tab:continuous_stationary_noiseless_summary} and \autoref{fig:noiseless_continuous_error} that the \gls{con_filter} tracks the Gazebo ground truth odometry almost perfectly. The maximum horizontal error is $2\times 10^{-6}$ m and the maximum yaw error is $4\times10^{-5}$ degrees. These errors would be completely indistinguishable from perfect accuracy in the real world.

\subsection{Mobile Robot}

\end{figure}
\section{Noisy Individual Operation}
\section{Noisy Group Operation}
\end{document}