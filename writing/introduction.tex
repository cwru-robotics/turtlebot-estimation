\documentclass[thesis.tex]{subfile}
\begin{document}
\chapter{Introduction}
\section{Mobile Robot Localization}
Robotic navigation can be divided into three questions: "Where am I?", "Where am I going?", and "How should I get there?" \cite{Leonard1991}. The first of these questions, "Where am I?" is the problem of localization. This second and third are problems of goal-setting and path-planning. All three are complex issues, but the focus of this thesis will be on localization. Determining where a robot is in reference to its surroundings is a fundamental precursor to all other robotic motion problems.

The main idea behind localization is combining sensor measurements that assess the state of the robot with measurements that represent the state of the surroundings \cite{Roumeliotis2002}. There are numerous methods for doing this, and new methods and improvements on existing methods are constantly appearing. Most of these methods vary in their chosen estimator and how they filter out inherent noise in the measurements. This thesis will explore some of the existing methods, but the chosen method is the \gls{ukf} \cite{Julier1997}.

How these sensor measurements used for localization are obtained is equally as important as the type of estimator used for localizing the robot. Adding more sensors into the state estimate will generally increase the accuracy. The typical approach has been to add more sensors onto an individual robot in order to increase the accuracy of its location estimate. However this thesis explores a distributed system for sharing information between different robots and examines the possible increase in accuracy that this can bring.

\section{Robotic Security}
   

\section{Problem Statement}
Now, I specify the exact localization problem that this thesis addresses. Our system contains the following assumptions and constraints.

\begin{enumerate}
\item $M$ independent mobile robots operating in a planar environment.
\item Wireless communication abilities between all robots in the environment.
\item No known map of environment, but GPS for localization in a world frame is available.
\item Localization must be able to seamlessly transition between operating with and without pose estimates from other robots.
\item Minimize complexity and added computations to Extended Kalman Filters used for localization.
\end{enumerate}
\end{document}